\documentclass[10pt]{article}

%opening
\title{
	{A CONCEPT PAPER ON  CAMPUS VIOLENCE AROUND MAKERERE UNIVERSITY}\\
}
\author{
{KABUUNGA HAMIDU 215004244   15/U/5389/PS}
}
\begin{document}
\title{	{A CONCEPT PAPER ON  CAMPUS VIOLENCE AROUND MAKERERE UNIVERSITY}\\}
\maketitle

\

\

\section{INTRODUCTION}
{The principal objective of this study is to examine the issue of campus violence inside the great gates of Makerere 
	University.Violence can be defined as the intentional use of physical force or power, threatened or actual, against oneself, another person, or against a group or community, that either results in or has a high likelihood of resulting in injury, death, psychological harm and maldevelopment. Makerere university is porous and vulnerable to various types of threats for instance stalking,domestic violence,sexual violence and assault and other activities conducted by disgruntled students or employees from both internal and external sources.I truly believe that for a campus community to be truly healthy, it must be guided by the values of multicultural inclusion, respect, and equality. Intolerance has no place at an institution of higher learning.}
\section{BACKGROUND TO THE PROBLEM}
{The country's oldest university has witnessed incidents of campus violence including suicide, sexual violence and sexual assault, thefts and robberies ranging from removal of car parts from car parks, computers or computer accessories, phones, cameras, laptops to students’ property from halls of residence.Other incidents include cyber crimes including sending of threatening anonymous text messages or emails, or sending pornographic pictures on phones or social networks. On the issue of sexual violence in particular, based on the 2004 findings on Situational Analysis of Gender Terrain at Makerere University, conducted by the Gender Mainstreaming Division,a Sexual Harassment Prevention Policy was established. In line with the constitution of the Republic of Uganda that guarantees all Ugandans equality, dignity and non-discrimination,the University pledged to commit to respect the rights of all the students and academic staff. And this is still on-going with the potential to track success over time.
}
\section{PROBLEM STATEMENT}
{My study mainly focuses on campus violence in a perspective of a unique population around Makerere university.The purpose of this research is to confront this serious issue of campus violence through analyzing campus violence patterns, types of violence, methodological problems with collecting campus violence data, underlying issues related to campus violence and promising practices to prevent and address campus violence.
}

\section{OBJECTIVES}
\subsection{Main objective}
{The main objective of this research is to examine the issue of campus violence in Makerere University.}
\subsection{Specific objectives}
{   To findout the magnitude or level of violence around campus in Makerere university.\\
	To analyze the university's plan for strategic and practical intervention towards addressing violence around campus.\\
	To see how this research can stimulate positive discussions on issues regarding  violence around Makerere university.\\ 
	To compare the trend of campus violence in Makerere and other universities in the region and abroad.
}
\section{METHODOLOGY}
{I intend to use the electronic data collection forms, interviews and review literation of the collected data from various sources about the growing syndrome of  campus violence in most universities around the world. The collected data will be subjected to analysis to ensure consistency and efficiency.The data to be collected will include images of interviewed people as well as recordings of conversations between me and various students i interact with around campus inorder to keep track of information from first hand sources. I will use ODK build to create electronic data collection forms which i wiil upload to my phone to be used during data collection through ODK collect installed on my Android device. The collected data will be stored on ODK aggregate server configured on my computer and will be fully analyzed, assessed and stored to facilitate future research and findings on campus violence in Makerere university.}
\section{OUTCOMES}
{I highly anticipate to get the best results out of this research that I am going to undertake. As clearly specified I will collect all the data using ODK collect on my android device and the data will be stored on the aggregate server. The outcomes of this research will facilitate further recommendations and suggestions on how to curb the vice out of the great gates of Makerere university.
}
\section{REFERENCES}
{Allan, E. (n.d.). Examining and transforming
	campus hazing cultures. Retrieved April 27,
	2004, from http://www.stophazing.org/
	hazingstudy.htm
Epstein, J. (2000). Hate crimes and other criminal
acts on campus. Retrieved January 21, 2004,
from http://www.campussafety.org/
information/artop/jepstein01.html
Makerere University, Policy and Regulations on Sexual Harassment Prevention (2006)}

\end{document}


